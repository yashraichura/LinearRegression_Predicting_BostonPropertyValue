\documentclass[]{article}
\usepackage{lmodern}
\usepackage{amssymb,amsmath}
\usepackage{ifxetex,ifluatex}
\usepackage{fixltx2e} % provides \textsubscript
\ifnum 0\ifxetex 1\fi\ifluatex 1\fi=0 % if pdftex
  \usepackage[T1]{fontenc}
  \usepackage[utf8]{inputenc}
\else % if luatex or xelatex
  \ifxetex
    \usepackage{mathspec}
  \else
    \usepackage{fontspec}
  \fi
  \defaultfontfeatures{Ligatures=TeX,Scale=MatchLowercase}
\fi
% use upquote if available, for straight quotes in verbatim environments
\IfFileExists{upquote.sty}{\usepackage{upquote}}{}
% use microtype if available
\IfFileExists{microtype.sty}{%
\usepackage{microtype}
\UseMicrotypeSet[protrusion]{basicmath} % disable protrusion for tt fonts
}{}
\usepackage[margin=1in]{geometry}
\usepackage{hyperref}
\hypersetup{unicode=true,
            pdftitle={Regression},
            pdfauthor={Purba Roy},
            pdfborder={0 0 0},
            breaklinks=true}
\urlstyle{same}  % don't use monospace font for urls
\usepackage{color}
\usepackage{fancyvrb}
\newcommand{\VerbBar}{|}
\newcommand{\VERB}{\Verb[commandchars=\\\{\}]}
\DefineVerbatimEnvironment{Highlighting}{Verbatim}{commandchars=\\\{\}}
% Add ',fontsize=\small' for more characters per line
\usepackage{framed}
\definecolor{shadecolor}{RGB}{248,248,248}
\newenvironment{Shaded}{\begin{snugshade}}{\end{snugshade}}
\newcommand{\AlertTok}[1]{\textcolor[rgb]{0.94,0.16,0.16}{#1}}
\newcommand{\AnnotationTok}[1]{\textcolor[rgb]{0.56,0.35,0.01}{\textbf{\textit{#1}}}}
\newcommand{\AttributeTok}[1]{\textcolor[rgb]{0.77,0.63,0.00}{#1}}
\newcommand{\BaseNTok}[1]{\textcolor[rgb]{0.00,0.00,0.81}{#1}}
\newcommand{\BuiltInTok}[1]{#1}
\newcommand{\CharTok}[1]{\textcolor[rgb]{0.31,0.60,0.02}{#1}}
\newcommand{\CommentTok}[1]{\textcolor[rgb]{0.56,0.35,0.01}{\textit{#1}}}
\newcommand{\CommentVarTok}[1]{\textcolor[rgb]{0.56,0.35,0.01}{\textbf{\textit{#1}}}}
\newcommand{\ConstantTok}[1]{\textcolor[rgb]{0.00,0.00,0.00}{#1}}
\newcommand{\ControlFlowTok}[1]{\textcolor[rgb]{0.13,0.29,0.53}{\textbf{#1}}}
\newcommand{\DataTypeTok}[1]{\textcolor[rgb]{0.13,0.29,0.53}{#1}}
\newcommand{\DecValTok}[1]{\textcolor[rgb]{0.00,0.00,0.81}{#1}}
\newcommand{\DocumentationTok}[1]{\textcolor[rgb]{0.56,0.35,0.01}{\textbf{\textit{#1}}}}
\newcommand{\ErrorTok}[1]{\textcolor[rgb]{0.64,0.00,0.00}{\textbf{#1}}}
\newcommand{\ExtensionTok}[1]{#1}
\newcommand{\FloatTok}[1]{\textcolor[rgb]{0.00,0.00,0.81}{#1}}
\newcommand{\FunctionTok}[1]{\textcolor[rgb]{0.00,0.00,0.00}{#1}}
\newcommand{\ImportTok}[1]{#1}
\newcommand{\InformationTok}[1]{\textcolor[rgb]{0.56,0.35,0.01}{\textbf{\textit{#1}}}}
\newcommand{\KeywordTok}[1]{\textcolor[rgb]{0.13,0.29,0.53}{\textbf{#1}}}
\newcommand{\NormalTok}[1]{#1}
\newcommand{\OperatorTok}[1]{\textcolor[rgb]{0.81,0.36,0.00}{\textbf{#1}}}
\newcommand{\OtherTok}[1]{\textcolor[rgb]{0.56,0.35,0.01}{#1}}
\newcommand{\PreprocessorTok}[1]{\textcolor[rgb]{0.56,0.35,0.01}{\textit{#1}}}
\newcommand{\RegionMarkerTok}[1]{#1}
\newcommand{\SpecialCharTok}[1]{\textcolor[rgb]{0.00,0.00,0.00}{#1}}
\newcommand{\SpecialStringTok}[1]{\textcolor[rgb]{0.31,0.60,0.02}{#1}}
\newcommand{\StringTok}[1]{\textcolor[rgb]{0.31,0.60,0.02}{#1}}
\newcommand{\VariableTok}[1]{\textcolor[rgb]{0.00,0.00,0.00}{#1}}
\newcommand{\VerbatimStringTok}[1]{\textcolor[rgb]{0.31,0.60,0.02}{#1}}
\newcommand{\WarningTok}[1]{\textcolor[rgb]{0.56,0.35,0.01}{\textbf{\textit{#1}}}}
\usepackage{graphicx,grffile}
\makeatletter
\def\maxwidth{\ifdim\Gin@nat@width>\linewidth\linewidth\else\Gin@nat@width\fi}
\def\maxheight{\ifdim\Gin@nat@height>\textheight\textheight\else\Gin@nat@height\fi}
\makeatother
% Scale images if necessary, so that they will not overflow the page
% margins by default, and it is still possible to overwrite the defaults
% using explicit options in \includegraphics[width, height, ...]{}
\setkeys{Gin}{width=\maxwidth,height=\maxheight,keepaspectratio}
\IfFileExists{parskip.sty}{%
\usepackage{parskip}
}{% else
\setlength{\parindent}{0pt}
\setlength{\parskip}{6pt plus 2pt minus 1pt}
}
\setlength{\emergencystretch}{3em}  % prevent overfull lines
\providecommand{\tightlist}{%
  \setlength{\itemsep}{0pt}\setlength{\parskip}{0pt}}
\setcounter{secnumdepth}{0}
% Redefines (sub)paragraphs to behave more like sections
\ifx\paragraph\undefined\else
\let\oldparagraph\paragraph
\renewcommand{\paragraph}[1]{\oldparagraph{#1}\mbox{}}
\fi
\ifx\subparagraph\undefined\else
\let\oldsubparagraph\subparagraph
\renewcommand{\subparagraph}[1]{\oldsubparagraph{#1}\mbox{}}
\fi

%%% Use protect on footnotes to avoid problems with footnotes in titles
\let\rmarkdownfootnote\footnote%
\def\footnote{\protect\rmarkdownfootnote}

%%% Change title format to be more compact
\usepackage{titling}

% Create subtitle command for use in maketitle
\providecommand{\subtitle}[1]{
  \posttitle{
    \begin{center}\large#1\end{center}
    }
}

\setlength{\droptitle}{-2em}

  \title{Regression}
    \pretitle{\vspace{\droptitle}\centering\huge}
  \posttitle{\par}
    \author{Purba Roy}
    \preauthor{\centering\large\emph}
  \postauthor{\par}
    \date{}
    \predate{}\postdate{}
  

\begin{document}
\maketitle

\begin{Shaded}
\begin{Highlighting}[]
\CommentTok{# Load standard libraries}
\KeywordTok{library}\NormalTok{(tidyverse)}
\KeywordTok{library}\NormalTok{(MASS) }\CommentTok{# Modern applied statistics functions}
\KeywordTok{library}\NormalTok{(ggplot2)}
\end{Highlighting}
\end{Shaded}

\textbf{Housing Values in Suburbs of Boston}

In this problem we will use the Boston dataset that is available in the
\texttt{MASS} package. This dataset contains information about median
house value for 506 neighborhoods in Boston, MA.

\begin{Shaded}
\begin{Highlighting}[]
\CommentTok{# Loading the data}
\NormalTok{BostonData <-}\StringTok{ }\NormalTok{Boston}
\end{Highlighting}
\end{Shaded}

\benum

\item

Data Description. Tidy data as necessary.

\textbf{There are 14 columns/ variables, and 506 rows of data.}
\textbf{The Boston Dataset has the following variables:} \textbf{crim:
this depicts the crime per capita rate in a town} \textbf{zn: this is
the proportion of residential land zone for lots that have an area more
than 25,000 sq ft} \textbf{indus: This depicts the proportion of
non-retail business acres per town} \textbf{chas: this is a dummy
variable for charles river where if the value is 1, the tract bounds
river} \textbf{nox : nitrogen oxides concentration} \textbf{rm : average
number of rooms per dwelling} \textbf{age: this is the proportion of
owner-occupied units built prior to 1940} \textbf{dis :This is the
weighted mean of distances to five Boston employment centres}
\textbf{rad: this is the index of accessibility to radial highways.}
\textbf{tax ; This is the full-value property-tax rate per \$10,000}
\textbf{ptratio: the pupil-teacher ratio by town } \textbf{black: This
is the proportion of blacks by town} \textbf{lstat: This is the lower
status of the population in percentage.} \textbf{medv: This is the
median value of owner-occupied homes in \$1000s}

\textbf{We see that the variables where the data is a whole number, the
data type is integer, and for other cases, its numeric. We convert chas
to categorical data as its value is only 0 and 1} \textbf{None of the
variables have NA data, hence its already a clean data.}

\begin{Shaded}
\begin{Highlighting}[]
\CommentTok{# to find the dimensions of the data set}
\KeywordTok{dim}\NormalTok{(BostonData)}
\end{Highlighting}
\end{Shaded}

\begin{verbatim}
## [1] 506  14
\end{verbatim}

\begin{Shaded}
\begin{Highlighting}[]
\CommentTok{# to check the variable names and respective data.}
\KeywordTok{head}\NormalTok{(BostonData)}
\end{Highlighting}
\end{Shaded}

\begin{verbatim}
##      crim zn indus chas   nox    rm  age    dis rad tax ptratio  black
## 1 0.00632 18  2.31    0 0.538 6.575 65.2 4.0900   1 296    15.3 396.90
## 2 0.02731  0  7.07    0 0.469 6.421 78.9 4.9671   2 242    17.8 396.90
## 3 0.02729  0  7.07    0 0.469 7.185 61.1 4.9671   2 242    17.8 392.83
## 4 0.03237  0  2.18    0 0.458 6.998 45.8 6.0622   3 222    18.7 394.63
## 5 0.06905  0  2.18    0 0.458 7.147 54.2 6.0622   3 222    18.7 396.90
## 6 0.02985  0  2.18    0 0.458 6.430 58.7 6.0622   3 222    18.7 394.12
##   lstat medv
## 1  4.98 24.0
## 2  9.14 21.6
## 3  4.03 34.7
## 4  2.94 33.4
## 5  5.33 36.2
## 6  5.21 28.7
\end{verbatim}

\begin{Shaded}
\begin{Highlighting}[]
\NormalTok{?Boston}
\end{Highlighting}
\end{Shaded}

\begin{verbatim}
## starting httpd help server ... done
\end{verbatim}

\begin{Shaded}
\begin{Highlighting}[]
\CommentTok{#View(BostonData)}

\CommentTok{# to check the datatype of each variable.}
\KeywordTok{sapply}\NormalTok{(BostonData,class)}
\end{Highlighting}
\end{Shaded}

\begin{verbatim}
##      crim        zn     indus      chas       nox        rm       age 
## "numeric" "numeric" "numeric" "integer" "numeric" "numeric" "numeric" 
##       dis       rad       tax   ptratio     black     lstat      medv 
## "numeric" "integer" "numeric" "numeric" "numeric" "numeric" "numeric"
\end{verbatim}

\begin{Shaded}
\begin{Highlighting}[]
\CommentTok{# converting chas  to categorical, as they depict categorical values, so that we can eliminate them from our linear regression model.}

\NormalTok{BostonData}\OperatorTok{$}\NormalTok{chas <-}\KeywordTok{as.factor}\NormalTok{(BostonData}\OperatorTok{$}\NormalTok{chas)}
\end{Highlighting}
\end{Shaded}

\textbf{I have taken medv as the response variable of interest as it is
a continuous numeric value, and variables such as tax, lstt, dis, etc
are dependant on the value of medv} \textbf{We see that our varible medv
ha sa normal distribution}

\begin{Shaded}
\begin{Highlighting}[]
\CommentTok{# to check the distribution of medv}
\KeywordTok{ggplot}\NormalTok{(BostonData) }\OperatorTok{+}
\StringTok{  }\KeywordTok{geom_density}\NormalTok{(}\KeywordTok{aes}\NormalTok{(}\DataTypeTok{x=}\NormalTok{medv), }\DataTypeTok{alpha=}\FloatTok{0.1}\NormalTok{)}
\end{Highlighting}
\end{Shaded}

\includegraphics{Boston_rmd_files/figure-latex/unnamed-chunk-3-1.pdf}

\textbf{ANSWER} \textbf{On getting the summary and residual plots of the
linear regression model, we see that the p- value of all the variables
are in the range of less than 2.2e-12 which is significantly than 0.05.
This says that all the variables are statistically significant}

\textbf{On Plotting the residual plot, we note the below observations:}

\textbf{better models for fitting the model} \textbf{For the predictor
variables lstat and rm, the residual points are scattered and do not
follow a trend, which makes it a good residual plot. Also, there is a
strong negative correlation for lstat (=-0.7376627), and a strong
positive corelation for rm (0.6953599), which proves that these 2
predictor variables are better for fitting the regression model.}

\begin{Shaded}
\begin{Highlighting}[]
 \CommentTok{#linearMod <- lm(BostonData$medv ~BostonData$indus, data=BostonData) }
 \CommentTok{#linearMod}
 \CommentTok{#residual <-resid(linearMod)}
 \CommentTok{#plot <- ggplot(data=data.frame(x=BostonData$medv, y= residual),}
  \CommentTok{#               aes(x=x,y=y)) + geom_point()+ geom_abline(slope=0,intercept=0)+stat_smooth(method="lm") }
  \CommentTok{#print(plot + ggtitle(i) +labs(x="medv", y="residual"))}

\CommentTok{# excluding chas as its an ordinal value and removing medv too.}
\NormalTok{column <-}\StringTok{ }\KeywordTok{colnames}\NormalTok{(BostonData)[}\KeywordTok{c}\NormalTok{(}\DecValTok{1}\OperatorTok{:}\DecValTok{3}\NormalTok{,}\DecValTok{5}\OperatorTok{:}\DecValTok{13}\NormalTok{)]}

\CommentTok{# initialising a dataframe for coefficients}
\NormalTok{coef_df <-}\StringTok{ }\KeywordTok{data.frame}\NormalTok{()}

\CommentTok{# creating a for loop to fit linear regression model for each predictor Vs Medv.}
\ControlFlowTok{for}\NormalTok{ (i }\ControlFlowTok{in}\NormalTok{ column)\{}
\NormalTok{  BostonDataSS <-}\StringTok{ }\KeywordTok{subset}\NormalTok{(BostonData, }\DataTypeTok{select=}\KeywordTok{c}\NormalTok{(}\StringTok{'medv'}\NormalTok{,i))}
\NormalTok{  linearMod <-}\StringTok{ }\KeywordTok{lm}\NormalTok{(medv }\OperatorTok{~}\StringTok{ }\NormalTok{.,BostonDataSS)}
  \KeywordTok{print}\NormalTok{(i)}
  \KeywordTok{print}\NormalTok{(}\KeywordTok{summary}\NormalTok{(linearMod))}
\NormalTok{  residual <-}\KeywordTok{resid}\NormalTok{(linearMod)}
\NormalTok{  plot <-}\StringTok{ }\KeywordTok{ggplot}\NormalTok{(}\DataTypeTok{data=}\KeywordTok{data.frame}\NormalTok{(}\DataTypeTok{x=}\NormalTok{BostonDataSS}\OperatorTok{$}\NormalTok{medv, }\DataTypeTok{y=}\NormalTok{ residual),}
                 \KeywordTok{aes}\NormalTok{(}\DataTypeTok{x=}\NormalTok{x,}\DataTypeTok{y=}\NormalTok{y)) }\OperatorTok{+}\StringTok{ }\KeywordTok{geom_point}\NormalTok{()}\OperatorTok{+}\StringTok{ }\KeywordTok{geom_abline}\NormalTok{(}\DataTypeTok{slope=}\DecValTok{0}\NormalTok{,}\DataTypeTok{intercept=}\DecValTok{0}\NormalTok{)}\OperatorTok{+}\KeywordTok{stat_smooth}\NormalTok{(}\DataTypeTok{method=}\StringTok{"lm"}\NormalTok{) }
  \KeywordTok{print}\NormalTok{(plot }\OperatorTok{+}\StringTok{ }\KeywordTok{ggtitle}\NormalTok{(i) }\OperatorTok{+}\KeywordTok{labs}\NormalTok{(}\DataTypeTok{x=}\StringTok{"medv"}\NormalTok{, }\DataTypeTok{y=}\StringTok{"residual"}\NormalTok{))}
  
\NormalTok{  coef_df <-}\StringTok{ }\KeywordTok{rbind}\NormalTok{(coef_df, }\KeywordTok{summary}\NormalTok{(linearMod)}\OperatorTok{$}\NormalTok{coef[i,}\StringTok{"Estimate"}\NormalTok{])}
\NormalTok{\}}
\end{Highlighting}
\end{Shaded}

\begin{verbatim}
## [1] "crim"
## 
## Call:
## lm(formula = medv ~ ., data = BostonDataSS)
## 
## Residuals:
##     Min      1Q  Median      3Q     Max 
## -16.957  -5.449  -2.007   2.512  29.800 
## 
## Coefficients:
##             Estimate Std. Error t value Pr(>|t|)    
## (Intercept) 24.03311    0.40914   58.74   <2e-16 ***
## crim        -0.41519    0.04389   -9.46   <2e-16 ***
## ---
## Signif. codes:  0 '***' 0.001 '**' 0.01 '*' 0.05 '.' 0.1 ' ' 1
## 
## Residual standard error: 8.484 on 504 degrees of freedom
## Multiple R-squared:  0.1508, Adjusted R-squared:  0.1491 
## F-statistic: 89.49 on 1 and 504 DF,  p-value: < 2.2e-16
\end{verbatim}

\includegraphics{Boston_rmd_files/figure-latex/unnamed-chunk-4-1.pdf}

\begin{verbatim}
## [1] "zn"
## 
## Call:
## lm(formula = medv ~ ., data = BostonDataSS)
## 
## Residuals:
##     Min      1Q  Median      3Q     Max 
## -15.918  -5.518  -1.006   2.757  29.082 
## 
## Coefficients:
##             Estimate Std. Error t value Pr(>|t|)    
## (Intercept) 20.91758    0.42474  49.248   <2e-16 ***
## zn           0.14214    0.01638   8.675   <2e-16 ***
## ---
## Signif. codes:  0 '***' 0.001 '**' 0.01 '*' 0.05 '.' 0.1 ' ' 1
## 
## Residual standard error: 8.587 on 504 degrees of freedom
## Multiple R-squared:  0.1299, Adjusted R-squared:  0.1282 
## F-statistic: 75.26 on 1 and 504 DF,  p-value: < 2.2e-16
\end{verbatim}

\includegraphics{Boston_rmd_files/figure-latex/unnamed-chunk-4-2.pdf}

\begin{verbatim}
## [1] "indus"
## 
## Call:
## lm(formula = medv ~ ., data = BostonDataSS)
## 
## Residuals:
##     Min      1Q  Median      3Q     Max 
## -13.017  -4.917  -1.457   3.180  32.943 
## 
## Coefficients:
##             Estimate Std. Error t value Pr(>|t|)    
## (Intercept) 29.75490    0.68345   43.54   <2e-16 ***
## indus       -0.64849    0.05226  -12.41   <2e-16 ***
## ---
## Signif. codes:  0 '***' 0.001 '**' 0.01 '*' 0.05 '.' 0.1 ' ' 1
## 
## Residual standard error: 8.057 on 504 degrees of freedom
## Multiple R-squared:  0.234,  Adjusted R-squared:  0.2325 
## F-statistic:   154 on 1 and 504 DF,  p-value: < 2.2e-16
\end{verbatim}

\includegraphics{Boston_rmd_files/figure-latex/unnamed-chunk-4-3.pdf}

\begin{verbatim}
## [1] "nox"
## 
## Call:
## lm(formula = medv ~ ., data = BostonDataSS)
## 
## Residuals:
##     Min      1Q  Median      3Q     Max 
## -13.691  -5.121  -2.161   2.959  31.310 
## 
## Coefficients:
##             Estimate Std. Error t value Pr(>|t|)    
## (Intercept)   41.346      1.811   22.83   <2e-16 ***
## nox          -33.916      3.196  -10.61   <2e-16 ***
## ---
## Signif. codes:  0 '***' 0.001 '**' 0.01 '*' 0.05 '.' 0.1 ' ' 1
## 
## Residual standard error: 8.323 on 504 degrees of freedom
## Multiple R-squared:  0.1826, Adjusted R-squared:  0.181 
## F-statistic: 112.6 on 1 and 504 DF,  p-value: < 2.2e-16
\end{verbatim}

\includegraphics{Boston_rmd_files/figure-latex/unnamed-chunk-4-4.pdf}

\begin{verbatim}
## [1] "rm"
## 
## Call:
## lm(formula = medv ~ ., data = BostonDataSS)
## 
## Residuals:
##     Min      1Q  Median      3Q     Max 
## -23.346  -2.547   0.090   2.986  39.433 
## 
## Coefficients:
##             Estimate Std. Error t value Pr(>|t|)    
## (Intercept)  -34.671      2.650  -13.08   <2e-16 ***
## rm             9.102      0.419   21.72   <2e-16 ***
## ---
## Signif. codes:  0 '***' 0.001 '**' 0.01 '*' 0.05 '.' 0.1 ' ' 1
## 
## Residual standard error: 6.616 on 504 degrees of freedom
## Multiple R-squared:  0.4835, Adjusted R-squared:  0.4825 
## F-statistic: 471.8 on 1 and 504 DF,  p-value: < 2.2e-16
\end{verbatim}

\includegraphics{Boston_rmd_files/figure-latex/unnamed-chunk-4-5.pdf}

\begin{verbatim}
## [1] "age"
## 
## Call:
## lm(formula = medv ~ ., data = BostonDataSS)
## 
## Residuals:
##     Min      1Q  Median      3Q     Max 
## -15.097  -5.138  -1.958   2.397  31.338 
## 
## Coefficients:
##             Estimate Std. Error t value Pr(>|t|)    
## (Intercept) 30.97868    0.99911  31.006   <2e-16 ***
## age         -0.12316    0.01348  -9.137   <2e-16 ***
## ---
## Signif. codes:  0 '***' 0.001 '**' 0.01 '*' 0.05 '.' 0.1 ' ' 1
## 
## Residual standard error: 8.527 on 504 degrees of freedom
## Multiple R-squared:  0.1421, Adjusted R-squared:  0.1404 
## F-statistic: 83.48 on 1 and 504 DF,  p-value: < 2.2e-16
\end{verbatim}

\includegraphics{Boston_rmd_files/figure-latex/unnamed-chunk-4-6.pdf}

\begin{verbatim}
## [1] "dis"
## 
## Call:
## lm(formula = medv ~ ., data = BostonDataSS)
## 
## Residuals:
##     Min      1Q  Median      3Q     Max 
## -15.016  -5.556  -1.865   2.288  30.377 
## 
## Coefficients:
##             Estimate Std. Error t value Pr(>|t|)    
## (Intercept)  18.3901     0.8174  22.499  < 2e-16 ***
## dis           1.0916     0.1884   5.795 1.21e-08 ***
## ---
## Signif. codes:  0 '***' 0.001 '**' 0.01 '*' 0.05 '.' 0.1 ' ' 1
## 
## Residual standard error: 8.914 on 504 degrees of freedom
## Multiple R-squared:  0.06246,    Adjusted R-squared:  0.0606 
## F-statistic: 33.58 on 1 and 504 DF,  p-value: 1.207e-08
\end{verbatim}

\includegraphics{Boston_rmd_files/figure-latex/unnamed-chunk-4-7.pdf}

\begin{verbatim}
## [1] "rad"
## 
## Call:
## lm(formula = medv ~ ., data = BostonDataSS)
## 
## Residuals:
##     Min      1Q  Median      3Q     Max 
## -17.770  -5.199  -1.967   3.321  33.292 
## 
## Coefficients:
##             Estimate Std. Error t value Pr(>|t|)    
## (Intercept) 26.38213    0.56176  46.964   <2e-16 ***
## rad         -0.40310    0.04349  -9.269   <2e-16 ***
## ---
## Signif. codes:  0 '***' 0.001 '**' 0.01 '*' 0.05 '.' 0.1 ' ' 1
## 
## Residual standard error: 8.509 on 504 degrees of freedom
## Multiple R-squared:  0.1456, Adjusted R-squared:  0.1439 
## F-statistic: 85.91 on 1 and 504 DF,  p-value: < 2.2e-16
\end{verbatim}

\includegraphics{Boston_rmd_files/figure-latex/unnamed-chunk-4-8.pdf}

\begin{verbatim}
## [1] "tax"
## 
## Call:
## lm(formula = medv ~ ., data = BostonDataSS)
## 
## Residuals:
##     Min      1Q  Median      3Q     Max 
## -14.091  -5.173  -2.085   3.158  34.058 
## 
## Coefficients:
##              Estimate Std. Error t value Pr(>|t|)    
## (Intercept) 32.970654   0.948296   34.77   <2e-16 ***
## tax         -0.025568   0.002147  -11.91   <2e-16 ***
## ---
## Signif. codes:  0 '***' 0.001 '**' 0.01 '*' 0.05 '.' 0.1 ' ' 1
## 
## Residual standard error: 8.133 on 504 degrees of freedom
## Multiple R-squared:  0.2195, Adjusted R-squared:  0.218 
## F-statistic: 141.8 on 1 and 504 DF,  p-value: < 2.2e-16
\end{verbatim}

\includegraphics{Boston_rmd_files/figure-latex/unnamed-chunk-4-9.pdf}

\begin{verbatim}
## [1] "ptratio"
## 
## Call:
## lm(formula = medv ~ ., data = BostonDataSS)
## 
## Residuals:
##      Min       1Q   Median       3Q      Max 
## -18.8342  -4.8262  -0.6426   3.1571  31.2303 
## 
## Coefficients:
##             Estimate Std. Error t value Pr(>|t|)    
## (Intercept)   62.345      3.029   20.58   <2e-16 ***
## ptratio       -2.157      0.163  -13.23   <2e-16 ***
## ---
## Signif. codes:  0 '***' 0.001 '**' 0.01 '*' 0.05 '.' 0.1 ' ' 1
## 
## Residual standard error: 7.931 on 504 degrees of freedom
## Multiple R-squared:  0.2578, Adjusted R-squared:  0.2564 
## F-statistic: 175.1 on 1 and 504 DF,  p-value: < 2.2e-16
\end{verbatim}

\includegraphics{Boston_rmd_files/figure-latex/unnamed-chunk-4-10.pdf}

\begin{verbatim}
## [1] "black"
## 
## Call:
## lm(formula = medv ~ ., data = BostonDataSS)
## 
## Residuals:
##     Min      1Q  Median      3Q     Max 
## -18.884  -4.862  -1.684   2.932  27.763 
## 
## Coefficients:
##              Estimate Std. Error t value Pr(>|t|)    
## (Intercept) 10.551034   1.557463   6.775 3.49e-11 ***
## black        0.033593   0.004231   7.941 1.32e-14 ***
## ---
## Signif. codes:  0 '***' 0.001 '**' 0.01 '*' 0.05 '.' 0.1 ' ' 1
## 
## Residual standard error: 8.679 on 504 degrees of freedom
## Multiple R-squared:  0.1112, Adjusted R-squared:  0.1094 
## F-statistic: 63.05 on 1 and 504 DF,  p-value: 1.318e-14
\end{verbatim}

\includegraphics{Boston_rmd_files/figure-latex/unnamed-chunk-4-11.pdf}

\begin{verbatim}
## [1] "lstat"
## 
## Call:
## lm(formula = medv ~ ., data = BostonDataSS)
## 
## Residuals:
##     Min      1Q  Median      3Q     Max 
## -15.168  -3.990  -1.318   2.034  24.500 
## 
## Coefficients:
##             Estimate Std. Error t value Pr(>|t|)    
## (Intercept) 34.55384    0.56263   61.41   <2e-16 ***
## lstat       -0.95005    0.03873  -24.53   <2e-16 ***
## ---
## Signif. codes:  0 '***' 0.001 '**' 0.01 '*' 0.05 '.' 0.1 ' ' 1
## 
## Residual standard error: 6.216 on 504 degrees of freedom
## Multiple R-squared:  0.5441, Adjusted R-squared:  0.5432 
## F-statistic: 601.6 on 1 and 504 DF,  p-value: < 2.2e-16
\end{verbatim}

\includegraphics{Boston_rmd_files/figure-latex/unnamed-chunk-4-12.pdf}

\begin{Shaded}
\begin{Highlighting}[]
\CommentTok{# find the correlation}
\KeywordTok{cor}\NormalTok{(BostonData}\OperatorTok{$}\NormalTok{medv, BostonData}\OperatorTok{$}\NormalTok{rm)}
\end{Highlighting}
\end{Shaded}

\begin{verbatim}
## [1] 0.6953599
\end{verbatim}

\begin{Shaded}
\begin{Highlighting}[]
\CommentTok{# find the correlation}
\KeywordTok{cor}\NormalTok{(BostonData}\OperatorTok{$}\NormalTok{medv, BostonData}\OperatorTok{$}\NormalTok{lstat)}
\end{Highlighting}
\end{Shaded}

\begin{verbatim}
## [1] -0.7376627
\end{verbatim}

\item

Fitting a multiple regression model to predict the response using all of
the predictors.

\textbf{For the predictor variables, age and indus, the p values are
much higher than 0.05 which mean that they are not statistically
significant. For the remaining variables, the p values are strongly
statiscally significant of the order 0.} \textbf{FOr predictor
variables, for which the p value are less than 0.05, we reject the null
hypothesis. SO we do not reject the null hypothesis for age and indus}
\textbf{So this means that indus and age behave like independant
variables.} \textbf{The multiple R square value is 0.7355}

\begin{Shaded}
\begin{Highlighting}[]
\NormalTok{BostonDataO <-}\StringTok{ }\NormalTok{BostonData[}\OperatorTok{-}\DecValTok{4}\NormalTok{]}
\KeywordTok{options}\NormalTok{(}\DataTypeTok{scipen=}\DecValTok{999}\NormalTok{)}
\NormalTok{multiple_reg <-}\StringTok{ }\KeywordTok{lm}\NormalTok{(medv }\OperatorTok{~}\NormalTok{., }\DataTypeTok{data=}\NormalTok{BostonDataO)}

\KeywordTok{summary}\NormalTok{(multiple_reg)}
\end{Highlighting}
\end{Shaded}

\begin{verbatim}
## 
## Call:
## lm(formula = medv ~ ., data = BostonDataO)
## 
## Residuals:
##      Min       1Q   Median       3Q      Max 
## -13.3968  -2.8103  -0.6455   1.9141  26.3755 
## 
## Coefficients:
##               Estimate Std. Error t value             Pr(>|t|)    
## (Intercept)  36.891960   5.146516   7.168    0.000000000002790 ***
## crim         -0.113139   0.033113  -3.417             0.000686 ***
## zn            0.047052   0.013847   3.398             0.000734 ***
## indus         0.040311   0.061707   0.653             0.513889    
## nox         -17.366999   3.851224  -4.509    0.000008130111167 ***
## rm            3.850492   0.421402   9.137 < 0.0000000000000002 ***
## age           0.002784   0.013309   0.209             0.834407    
## dis          -1.485374   0.201187  -7.383    0.000000000000664 ***
## rad           0.328311   0.066542   4.934    0.000001104312049 ***
## tax          -0.013756   0.003766  -3.653             0.000287 ***
## ptratio      -0.990958   0.131399  -7.542    0.000000000000225 ***
## black         0.009741   0.002706   3.600             0.000351 ***
## lstat        -0.534158   0.051072 -10.459 < 0.0000000000000002 ***
## ---
## Signif. codes:  0 '***' 0.001 '**' 0.01 '*' 0.05 '.' 0.1 ' ' 1
## 
## Residual standard error: 4.787 on 493 degrees of freedom
## Multiple R-squared:  0.7355, Adjusted R-squared:  0.7291 
## F-statistic: 114.3 on 12 and 493 DF,  p-value: < 0.00000000000000022
\end{verbatim}

\begin{Shaded}
\begin{Highlighting}[]
\CommentTok{# for coefficients}
\NormalTok{multiple_reg}\OperatorTok{$}\NormalTok{coefficients}
\end{Highlighting}
\end{Shaded}

\begin{verbatim}
##   (Intercept)          crim            zn         indus           nox 
##  36.891959797  -0.113139078   0.047052458   0.040311454 -17.366999394 
##            rm           age           dis           rad           tax 
##   3.850491692   0.002783757  -1.485373898   0.328311011  -0.013755829 
##       ptratio         black         lstat 
##  -0.990958031   0.009741451  -0.534157620
\end{verbatim}

\item

Creating a plot displaying the univariate regression coefficients on the
x-axis and the multiple regression coefficients on the y-axis.

\textbf{We plotted the scatter plot with multiple regression on the y
axis and linear regression on the x axis, and see the p value for the
different predictor variables.} \textbf{We see that for rm, the p value
for linear model was \textasciitilde{}10, and after multiple regression,
where we add other predictor variables to our model, the p value drops
to \textasciitilde{}3.} \textbf{For some of the predictor variables, the
p value of the linear and the multiple regresiion models are similar.}
\textbf{This shows that when we add random values, the effect of the
predictor on the response variables decreases the p value.}

\begin{Shaded}
\begin{Highlighting}[]
\CommentTok{# dropping the intercept value}
\NormalTok{Dropmultiple <-}\StringTok{ }\NormalTok{multiple_reg}\OperatorTok{$}\NormalTok{coefficients[}\OperatorTok{-}\DecValTok{1}\NormalTok{]}

\CommentTok{#converting to a data frame}
\NormalTok{multiple_reg_model <-}\StringTok{ }\KeywordTok{as.data.frame}\NormalTok{(Dropmultiple)}
\NormalTok{multiple_reg_model<-tibble}\OperatorTok{::}\KeywordTok{rownames_to_column}\NormalTok{(multiple_reg_model,}\StringTok{"linear_reg_model_old"}\NormalTok{)}
\NormalTok{multiple_reg_model}
\end{Highlighting}
\end{Shaded}

\begin{verbatim}
##    linear_reg_model_old  Dropmultiple
## 1                  crim  -0.113139078
## 2                    zn   0.047052458
## 3                 indus   0.040311454
## 4                   nox -17.366999394
## 5                    rm   3.850491692
## 6                   age   0.002783757
## 7                   dis  -1.485373898
## 8                   rad   0.328311011
## 9                   tax  -0.013755829
## 10              ptratio  -0.990958031
## 11                black   0.009741451
## 12                lstat  -0.534157620
\end{verbatim}

\begin{Shaded}
\begin{Highlighting}[]
\NormalTok{linear_reg_model_coef <-}\StringTok{ }\NormalTok{coef_df[,]}

\NormalTok{linear_reg_model_old <-}\StringTok{ }\KeywordTok{c}\NormalTok{(}\StringTok{"crim"}\NormalTok{,}\StringTok{"zn"}\NormalTok{,}\StringTok{"indus"}\NormalTok{,}\StringTok{"nox"}\NormalTok{,}\StringTok{"rm"}\NormalTok{,}\StringTok{"age"}\NormalTok{,}\StringTok{"dis"}\NormalTok{,}\StringTok{"rad"}\NormalTok{,}\StringTok{"tax"}\NormalTok{,}\StringTok{"ptratio"}\NormalTok{,}\StringTok{"black"}\NormalTok{,}\StringTok{"lstat"}\NormalTok{)}
\NormalTok{linear_reg_model_old<-}\KeywordTok{as.data.frame}\NormalTok{(linear_reg_model_old)}
\NormalTok{linear_reg_model<-}\KeywordTok{cbind}\NormalTok{(linear_reg_model_old,linear_reg_model_coef)}
\CommentTok{#linear_reg_model<-tibble::rownames_to_column(linear_reg_model,"Variables_Predictor")}

\NormalTok{dataset<-}\KeywordTok{left_join}\NormalTok{(linear_reg_model,multiple_reg_model, }\DataTypeTok{by=}\StringTok{"linear_reg_model_old"}\NormalTok{)}
\end{Highlighting}
\end{Shaded}

\begin{verbatim}
## Warning: Column `linear_reg_model_old` joining factor and character vector,
## coercing into character vector
\end{verbatim}

\begin{Shaded}
\begin{Highlighting}[]
\NormalTok{linear_reg_model}\OperatorTok{$}\NormalTok{model_type=}\StringTok{"one"}
\NormalTok{multiple_reg_model}\OperatorTok{$}\NormalTok{model_type=}\StringTok{"many"}
\KeywordTok{colnames}\NormalTok{(linear_reg_model)<-}\KeywordTok{c}\NormalTok{(}\StringTok{"linear_reg_model_old"}\NormalTok{,}\StringTok{"Dropmultiple"}\NormalTok{,}\StringTok{"model_type"}\NormalTok{)}
\NormalTok{final_dataset<-}\KeywordTok{rbind}\NormalTok{(linear_reg_model,multiple_reg_model)}

\CommentTok{# plotting}
\KeywordTok{ggplot}\NormalTok{(}\DataTypeTok{data=}\NormalTok{final_dataset, }\KeywordTok{aes}\NormalTok{(}\DataTypeTok{x=}\NormalTok{linear_reg_model_old, }\DataTypeTok{y=}\NormalTok{Dropmultiple, }\DataTypeTok{color=}\NormalTok{model_type))}\OperatorTok{+}\KeywordTok{geom_point}\NormalTok{()}
\end{Highlighting}
\end{Shaded}

\includegraphics{Boston_rmd_files/figure-latex/unnamed-chunk-6-1.pdf}

\begin{Shaded}
\begin{Highlighting}[]
\CommentTok{#plot(linear_reg_model,newTable, xlim = c(-5,15), ylim=c(-3,5))}
\end{Highlighting}
\end{Shaded}

\textbf{After comparing the different p values for different predictor
variables, we see that for a few, the avlues have increased.} \textbf{It
is uncertain that there is a non linear relation. SO a non linear
association might not be best for the dataset }

\begin{Shaded}
\begin{Highlighting}[]
\NormalTok{Poly_function <-}\StringTok{ }\ControlFlowTok{function}\NormalTok{(column)}
\NormalTok{\{}
\NormalTok{  poly <-}\StringTok{ }\KeywordTok{lm}\NormalTok{(medv }\OperatorTok{~}\StringTok{ }\KeywordTok{poly}\NormalTok{(column,}\DecValTok{3}\NormalTok{), }\DataTypeTok{data=}\NormalTok{ BostonData)}
\NormalTok{\}}

\NormalTok{Poly_crim <-}\StringTok{ }\KeywordTok{Poly_function}\NormalTok{(BostonData}\OperatorTok{$}\NormalTok{crim)}
\NormalTok{Poly_zn <-}\StringTok{ }\KeywordTok{Poly_function}\NormalTok{(BostonData}\OperatorTok{$}\NormalTok{zn)}
\NormalTok{Poly_indus <-}\StringTok{ }\KeywordTok{Poly_function}\NormalTok{(BostonData}\OperatorTok{$}\NormalTok{indus)}
\NormalTok{Poly_nox <-}\StringTok{ }\KeywordTok{Poly_function}\NormalTok{(BostonData}\OperatorTok{$}\NormalTok{nox)}
\NormalTok{Poly_rm <-}\StringTok{ }\KeywordTok{Poly_function}\NormalTok{(BostonData}\OperatorTok{$}\NormalTok{rm)}
\NormalTok{Poly_age <-}\StringTok{ }\KeywordTok{Poly_function}\NormalTok{(BostonData}\OperatorTok{$}\NormalTok{age)}
\NormalTok{Poly_dis <-}\StringTok{ }\KeywordTok{Poly_function}\NormalTok{(BostonData}\OperatorTok{$}\NormalTok{dis)}
\NormalTok{Poly_rad <-}\StringTok{ }\KeywordTok{Poly_function}\NormalTok{(BostonData}\OperatorTok{$}\NormalTok{rad)}
\NormalTok{Poly_tax <-}\StringTok{ }\KeywordTok{Poly_function}\NormalTok{(BostonData}\OperatorTok{$}\NormalTok{tax)}
\NormalTok{Poly_ptratio <-}\StringTok{ }\KeywordTok{Poly_function}\NormalTok{(BostonData}\OperatorTok{$}\NormalTok{ptratio)}
\NormalTok{Poly_black <-}\StringTok{ }\KeywordTok{Poly_function}\NormalTok{(BostonData}\OperatorTok{$}\NormalTok{black)}
\NormalTok{Poly_lstat <-}\StringTok{ }\KeywordTok{Poly_function}\NormalTok{(BostonData}\OperatorTok{$}\NormalTok{lstat)}
\NormalTok{Poly_medv <-}\StringTok{ }\KeywordTok{Poly_function}\NormalTok{(BostonData}\OperatorTok{$}\NormalTok{medv)}
\KeywordTok{summary}\NormalTok{(Poly_crim)}
\end{Highlighting}
\end{Shaded}

\begin{verbatim}
## 
## Call:
## lm(formula = medv ~ poly(column, 3), data = BostonData)
## 
## Residuals:
##     Min      1Q  Median      3Q     Max 
## -17.983  -4.975  -1.940   2.881  33.391 
## 
## Coefficients:
##                  Estimate Std. Error t value             Pr(>|t|)    
## (Intercept)       22.5328     0.3627  62.124 < 0.0000000000000002 ***
## poly(column, 3)1 -80.2545     8.1589  -9.836 < 0.0000000000000002 ***
## poly(column, 3)2  50.2416     8.1589   6.158        0.00000000151 ***
## poly(column, 3)3 -18.2905     8.1589  -2.242               0.0254 *  
## ---
## Signif. codes:  0 '***' 0.001 '**' 0.01 '*' 0.05 '.' 0.1 ' ' 1
## 
## Residual standard error: 8.159 on 502 degrees of freedom
## Multiple R-squared:  0.2177, Adjusted R-squared:  0.213 
## F-statistic: 46.57 on 3 and 502 DF,  p-value: < 0.00000000000000022
\end{verbatim}

\begin{Shaded}
\begin{Highlighting}[]
\KeywordTok{summary}\NormalTok{(Poly_zn)}
\end{Highlighting}
\end{Shaded}

\begin{verbatim}
## 
## Call:
## lm(formula = medv ~ poly(column, 3), data = BostonData)
## 
## Residuals:
##     Min      1Q  Median      3Q     Max 
## -15.449  -5.549  -1.049   3.225  29.551 
## 
## Coefficients:
##                  Estimate Std. Error t value             Pr(>|t|)    
## (Intercept)       22.5328     0.3747  60.129 < 0.0000000000000002 ***
## poly(column, 3)1  74.4966     8.4296   8.837 < 0.0000000000000002 ***
## poly(column, 3)2 -19.2591     8.4296  -2.285               0.0227 *  
## poly(column, 3)3  33.5309     8.4296   3.978            0.0000798 ***
## ---
## Signif. codes:  0 '***' 0.001 '**' 0.01 '*' 0.05 '.' 0.1 ' ' 1
## 
## Residual standard error: 8.43 on 502 degrees of freedom
## Multiple R-squared:  0.1649, Adjusted R-squared:  0.1599 
## F-statistic: 33.05 on 3 and 502 DF,  p-value: < 0.00000000000000022
\end{verbatim}

\begin{Shaded}
\begin{Highlighting}[]
\KeywordTok{summary}\NormalTok{(Poly_indus)}
\end{Highlighting}
\end{Shaded}

\begin{verbatim}
## 
## Call:
## lm(formula = medv ~ poly(column, 3), data = BostonData)
## 
## Residuals:
##     Min      1Q  Median      3Q     Max 
## -15.760  -4.725  -1.009   2.932  32.038 
## 
## Coefficients:
##                  Estimate Std. Error t value             Pr(>|t|)    
## (Intercept)       22.5328     0.3487  64.614 < 0.0000000000000002 ***
## poly(column, 3)1 -99.9759     7.8445 -12.745 < 0.0000000000000002 ***
## poly(column, 3)2  38.5184     7.8445   4.910           0.00000123 ***
## poly(column, 3)3 -18.6140     7.8445  -2.373                0.018 *  
## ---
## Signif. codes:  0 '***' 0.001 '**' 0.01 '*' 0.05 '.' 0.1 ' ' 1
## 
## Residual standard error: 7.844 on 502 degrees of freedom
## Multiple R-squared:  0.2768, Adjusted R-squared:  0.2725 
## F-statistic: 64.06 on 3 and 502 DF,  p-value: < 0.00000000000000022
\end{verbatim}

\begin{Shaded}
\begin{Highlighting}[]
\KeywordTok{summary}\NormalTok{(Poly_nox)}
\end{Highlighting}
\end{Shaded}

\begin{verbatim}
## 
## Call:
## lm(formula = medv ~ poly(column, 3), data = BostonData)
## 
## Residuals:
##     Min      1Q  Median      3Q     Max 
## -13.104  -5.020  -2.144   2.747  32.416 
## 
## Coefficients:
##                  Estimate Std. Error t value            Pr(>|t|)    
## (Intercept)       22.5328     0.3682  61.199 <0.0000000000000002 ***
## poly(column, 3)1 -88.3183     8.2823 -10.664 <0.0000000000000002 ***
## poly(column, 3)2  13.8989     8.2823   1.678              0.0939 .  
## poly(column, 3)3  16.9686     8.2823   2.049              0.0410 *  
## ---
## Signif. codes:  0 '***' 0.001 '**' 0.01 '*' 0.05 '.' 0.1 ' ' 1
## 
## Residual standard error: 8.282 on 502 degrees of freedom
## Multiple R-squared:  0.1939, Adjusted R-squared:  0.189 
## F-statistic: 40.24 on 3 and 502 DF,  p-value: < 0.00000000000000022
\end{verbatim}

\begin{Shaded}
\begin{Highlighting}[]
\KeywordTok{summary}\NormalTok{(Poly_age)}
\end{Highlighting}
\end{Shaded}

\begin{verbatim}
## 
## Call:
## lm(formula = medv ~ poly(column, 3), data = BostonData)
## 
## Residuals:
##     Min      1Q  Median      3Q     Max 
## -16.443  -4.909  -2.234   2.185  32.944 
## 
## Coefficients:
##                  Estimate Std. Error t value            Pr(>|t|)    
## (Intercept)       22.5328     0.3766  59.830 <0.0000000000000002 ***
## poly(column, 3)1 -77.9087     8.4717  -9.196 <0.0000000000000002 ***
## poly(column, 3)2 -23.3290     8.4717  -2.754              0.0061 ** 
## poly(column, 3)3  -8.6148     8.4717  -1.017              0.3097    
## ---
## Signif. codes:  0 '***' 0.001 '**' 0.01 '*' 0.05 '.' 0.1 ' ' 1
## 
## Residual standard error: 8.472 on 502 degrees of freedom
## Multiple R-squared:  0.1566, Adjusted R-squared:  0.1515 
## F-statistic: 31.06 on 3 and 502 DF,  p-value: < 0.00000000000000022
\end{verbatim}

\begin{Shaded}
\begin{Highlighting}[]
\KeywordTok{summary}\NormalTok{(Poly_dis)}
\end{Highlighting}
\end{Shaded}

\begin{verbatim}
## 
## Call:
## lm(formula = medv ~ poly(column, 3), data = BostonData)
## 
## Residuals:
##     Min      1Q  Median      3Q     Max 
## -12.571  -5.242  -2.037   2.397  34.769 
## 
## Coefficients:
##                  Estimate Std. Error t value             Pr(>|t|)    
## (Intercept)       22.5328     0.3879  58.082 < 0.0000000000000002 ***
## poly(column, 3)1  51.6551     8.7267   5.919          0.000000006 ***
## poly(column, 3)2 -37.5859     8.7267  -4.307          0.000019913 ***
## poly(column, 3)3  20.1322     8.7267   2.307               0.0215 *  
## ---
## Signif. codes:  0 '***' 0.001 '**' 0.01 '*' 0.05 '.' 0.1 ' ' 1
## 
## Residual standard error: 8.727 on 502 degrees of freedom
## Multiple R-squared:  0.105,  Adjusted R-squared:  0.09968 
## F-statistic: 19.64 on 3 and 502 DF,  p-value: 0.000000000004736
\end{verbatim}

\begin{Shaded}
\begin{Highlighting}[]
\KeywordTok{summary}\NormalTok{(Poly_rad)}
\end{Highlighting}
\end{Shaded}

\begin{verbatim}
## 
## Call:
## lm(formula = medv ~ poly(column, 3), data = BostonData)
## 
## Residuals:
##     Min      1Q  Median      3Q     Max 
## -16.630  -5.151  -2.017   3.169  33.594 
## 
## Coefficients:
##                  Estimate Std. Error t value             Pr(>|t|)    
## (Intercept)       22.5328     0.3721  60.557 < 0.0000000000000002 ***
## poly(column, 3)1 -78.8742     8.3700  -9.423 < 0.0000000000000002 ***
## poly(column, 3)2 -21.4799     8.3700  -2.566             0.010568 *  
## poly(column, 3)3 -29.4095     8.3700  -3.514             0.000482 ***
## ---
## Signif. codes:  0 '***' 0.001 '**' 0.01 '*' 0.05 '.' 0.1 ' ' 1
## 
## Residual standard error: 8.37 on 502 degrees of freedom
## Multiple R-squared:  0.1767, Adjusted R-squared:  0.1718 
## F-statistic: 35.91 on 3 and 502 DF,  p-value: < 0.00000000000000022
\end{verbatim}

\begin{Shaded}
\begin{Highlighting}[]
\KeywordTok{summary}\NormalTok{(Poly_tax)}
\end{Highlighting}
\end{Shaded}

\begin{verbatim}
## 
## Call:
## lm(formula = medv ~ poly(column, 3), data = BostonData)
## 
## Residuals:
##     Min      1Q  Median      3Q     Max 
## -15.109  -4.952  -1.878   2.957  33.694 
## 
## Coefficients:
##                  Estimate Std. Error t value            Pr(>|t|)    
## (Intercept)       22.5328     0.3608  62.460 <0.0000000000000002 ***
## poly(column, 3)1 -96.8366     8.1150 -11.933 <0.0000000000000002 ***
## poly(column, 3)2  14.9703     8.1150   1.845              0.0657 .  
## poly(column, 3)3  -7.5431     8.1150  -0.930              0.3531    
## ---
## Signif. codes:  0 '***' 0.001 '**' 0.01 '*' 0.05 '.' 0.1 ' ' 1
## 
## Residual standard error: 8.115 on 502 degrees of freedom
## Multiple R-squared:  0.2261, Adjusted R-squared:  0.2215 
## F-statistic: 48.89 on 3 and 502 DF,  p-value: < 0.00000000000000022
\end{verbatim}

\begin{Shaded}
\begin{Highlighting}[]
\KeywordTok{summary}\NormalTok{(Poly_ptratio)}
\end{Highlighting}
\end{Shaded}

\begin{verbatim}
## 
## Call:
## lm(formula = medv ~ poly(column, 3), data = BostonData)
## 
## Residuals:
##      Min       1Q   Median       3Q      Max 
## -17.7795  -5.0364  -0.9778   3.4766  31.1636 
## 
## Coefficients:
##                   Estimate Std. Error t value            Pr(>|t|)    
## (Intercept)        22.5328     0.3511  64.173 <0.0000000000000002 ***
## poly(column, 3)1 -104.9490     7.8984 -13.287 <0.0000000000000002 ***
## poly(column, 3)2  -12.6952     7.8984  -1.607               0.109    
## poly(column, 3)3  -14.9472     7.8984  -1.892               0.059 .  
## ---
## Signif. codes:  0 '***' 0.001 '**' 0.01 '*' 0.05 '.' 0.1 ' ' 1
## 
## Residual standard error: 7.898 on 502 degrees of freedom
## Multiple R-squared:  0.2669, Adjusted R-squared:  0.2625 
## F-statistic: 60.91 on 3 and 502 DF,  p-value: < 0.00000000000000022
\end{verbatim}

\begin{Shaded}
\begin{Highlighting}[]
\KeywordTok{summary}\NormalTok{(Poly_black)}
\end{Highlighting}
\end{Shaded}

\begin{verbatim}
## 
## Call:
## lm(formula = medv ~ poly(column, 3), data = BostonData)
## 
## Residuals:
##     Min      1Q  Median      3Q     Max 
## -19.005  -4.802  -1.613   2.852  28.051 
## 
## Coefficients:
##                  Estimate Std. Error t value             Pr(>|t|)    
## (Intercept)       22.5328     0.3861  58.360 < 0.0000000000000002 ***
## poly(column, 3)1  68.9194     8.6851   7.935   0.0000000000000138 ***
## poly(column, 3)2   9.1467     8.6851   1.053                0.293    
## poly(column, 3)3  -4.0541     8.6851  -0.467                0.641    
## ---
## Signif. codes:  0 '***' 0.001 '**' 0.01 '*' 0.05 '.' 0.1 ' ' 1
## 
## Residual standard error: 8.685 on 502 degrees of freedom
## Multiple R-squared:  0.1135, Adjusted R-squared:  0.1082 
## F-statistic: 21.43 on 3 and 502 DF,  p-value: 0.0000000000004463
\end{verbatim}

\begin{Shaded}
\begin{Highlighting}[]
\KeywordTok{summary}\NormalTok{(Poly_lstat)}
\end{Highlighting}
\end{Shaded}

\begin{verbatim}
## 
## Call:
## lm(formula = medv ~ poly(column, 3), data = BostonData)
## 
## Residuals:
##      Min       1Q   Median       3Q      Max 
## -14.5441  -3.7122  -0.5145   2.4846  26.4153 
## 
## Coefficients:
##                   Estimate Std. Error t value             Pr(>|t|)    
## (Intercept)        22.5328     0.2399  93.937 < 0.0000000000000002 ***
## poly(column, 3)1 -152.4595     5.3958 -28.255 < 0.0000000000000002 ***
## poly(column, 3)2   64.2272     5.3958  11.903 < 0.0000000000000002 ***
## poly(column, 3)3  -27.0511     5.3958  -5.013          0.000000743 ***
## ---
## Signif. codes:  0 '***' 0.001 '**' 0.01 '*' 0.05 '.' 0.1 ' ' 1
## 
## Residual standard error: 5.396 on 502 degrees of freedom
## Multiple R-squared:  0.6578, Adjusted R-squared:  0.6558 
## F-statistic: 321.7 on 3 and 502 DF,  p-value: < 0.00000000000000022
\end{verbatim}

\begin{Shaded}
\begin{Highlighting}[]
\KeywordTok{summary}\NormalTok{(Poly_medv)}
\end{Highlighting}
\end{Shaded}

\begin{verbatim}
## Warning in summary.lm(Poly_medv): essentially perfect fit: summary may be
## unreliable
\end{verbatim}

\begin{verbatim}
## 
## Call:
## lm(formula = medv ~ poly(column, 3), data = BostonData)
## 
## Residuals:
##                   Min                    1Q                Median 
## -0.000000000000102888 -0.000000000000000240  0.000000000000000436 
##                    3Q                   Max 
##  0.000000000000000607  0.000000000000013651 
## 
## Coefficients:
##                                 Estimate              Std. Error
## (Intercept)       22.5328063241106697490   0.0000000000000002121
## poly(column, 3)1 206.6792089568270114341   0.0000000000000047715
## poly(column, 3)2   0.0000000000000055259   0.0000000000000047715
## poly(column, 3)3  -0.0000000000000001888   0.0000000000000047715
##                                 t value            Pr(>|t|)    
## (Intercept)      106226911690490880.000 <0.0000000000000002 ***
## poly(column, 3)1  43315247865381000.000 <0.0000000000000002 ***
## poly(column, 3)2                  1.158               0.247    
## poly(column, 3)3                 -0.040               0.968    
## ---
## Signif. codes:  0 '***' 0.001 '**' 0.01 '*' 0.05 '.' 0.1 ' ' 1
## 
## Residual standard error: 0.000000000000004772 on 502 degrees of freedom
## Multiple R-squared:      1,  Adjusted R-squared:      1 
## F-statistic: 6.254e+32 on 3 and 502 DF,  p-value: < 0.00000000000000022
\end{verbatim}

\[ Y = \beta_0 + \beta_1 X + \beta_2 X^2 + \beta_3 X^3 + \epsilon \]

\item

performing a stepwise model selection procedure to determine the bets
fit model.

\textbf{We observe that only 11 predictor variables are shown the
output, where age and indus got dropped. This shows that its a more
optimised model as there was age and indus were acting as independant
variables.} \textbf{The AIC value has decreased from 3035.512 for
multiple regression to 3031.997 for stepwise selection model, which
shows that this model is a better fit now.} \textbf{We can hence infer
that there is a backward propogation }

\begin{Shaded}
\begin{Highlighting}[]
\NormalTok{stepwise <-}\StringTok{ }\KeywordTok{stepAIC}\NormalTok{(multiple_reg, }\DataTypeTok{direction=}\StringTok{"both"}\NormalTok{, }\DataTypeTok{trace=} \OtherTok{FALSE}\NormalTok{)}
\NormalTok{stepwise}
\end{Highlighting}
\end{Shaded}

\begin{verbatim}
## 
## Call:
## lm(formula = medv ~ crim + zn + nox + rm + dis + rad + tax + 
##     ptratio + black + lstat, data = BostonDataO)
## 
## Coefficients:
## (Intercept)         crim           zn          nox           rm  
##    36.62031     -0.11406      0.04574    -16.46915      3.84464  
##         dis          rad          tax      ptratio        black  
##    -1.52610      0.31553     -0.01267     -0.97844      0.00973  
##       lstat  
##    -0.52810
\end{verbatim}

\begin{Shaded}
\begin{Highlighting}[]
\CommentTok{#aic value}
\KeywordTok{AIC}\NormalTok{(stepwise)}
\end{Highlighting}
\end{Shaded}

\begin{verbatim}
## [1] 3031.997
\end{verbatim}

\begin{Shaded}
\begin{Highlighting}[]
\KeywordTok{AIC}\NormalTok{(multiple_reg)}
\end{Highlighting}
\end{Shaded}

\begin{verbatim}
## [1] 3035.512
\end{verbatim}

\item

Evaluating the statistical assumptions in my regression analysis by
performing a basic analysis of model residuals and any unusual
observations.

\textbf{We see that upon backward propogation, the data lies near 0,
this means that many residues have been removed and the data has been
optimized.} \textbf{We also see some outliers at the end and at the
begining.}

\begin{Shaded}
\begin{Highlighting}[]
\NormalTok{stepwiseForward <-}\StringTok{ }\KeywordTok{stepAIC}\NormalTok{(multiple_reg, }\DataTypeTok{direction=}\StringTok{"forward"}\NormalTok{)}
\end{Highlighting}
\end{Shaded}

\begin{verbatim}
## Start:  AIC=1597.55
## medv ~ crim + zn + indus + nox + rm + age + dis + rad + tax + 
##     ptratio + black + lstat
\end{verbatim}

\begin{Shaded}
\begin{Highlighting}[]
\NormalTok{stepwiseForward}
\end{Highlighting}
\end{Shaded}

\begin{verbatim}
## 
## Call:
## lm(formula = medv ~ crim + zn + indus + nox + rm + age + dis + 
##     rad + tax + ptratio + black + lstat, data = BostonDataO)
## 
## Coefficients:
## (Intercept)         crim           zn        indus          nox  
##   36.891960    -0.113139     0.047052     0.040311   -17.366999  
##          rm          age          dis          rad          tax  
##    3.850492     0.002784    -1.485374     0.328311    -0.013756  
##     ptratio        black        lstat  
##   -0.990958     0.009741    -0.534158
\end{verbatim}

\begin{Shaded}
\begin{Highlighting}[]
\NormalTok{residForward <-}\StringTok{ }\KeywordTok{resid}\NormalTok{(stepwiseForward)}
\KeywordTok{plot}\NormalTok{(residForward)}
\end{Highlighting}
\end{Shaded}

\includegraphics{Boston_rmd_files/figure-latex/unnamed-chunk-9-1.pdf}
\textbf{Upon plotting the qqplot, there are a number of outliers in the
data in the begining and at the end.} \textbf{The main concern which I
feel are the outliers, due to which the data isnt perfectly normal. }
\textbf{We see that because of the outliers, there is a right tail which
is making it a right skewed distribution.}

\begin{Shaded}
\begin{Highlighting}[]
\CommentTok{#ggplot(data=residForward) +geom_point() +geom_smooth()}
\KeywordTok{qqnorm}\NormalTok{(residForward)}
\end{Highlighting}
\end{Shaded}

\includegraphics{Boston_rmd_files/figure-latex/unnamed-chunk-10-1.pdf}

\eenum


\end{document}
